\section{Príručky pre programátora a používateľa}\label{sec:helpers}
\subsection{Príručka pre implementáciu umelej inteligencie}\label{subsec:helpers-ai}
Ako bolo vyššie spomenuté, časť aplikácie, ktorá je zameraná na umelú inteligenciu, je implementovaná v jazyku python.
Interakcia s touto časťou je vykonávaná prostredníctvom rozhrania príkazového riadku (skr. CLI, angl. command line
interface), pričom pre použitie je potrebné mať nainštalovaný interpreter jazyka python.
Znamená to, že program nie je skompilovaný do jedného binárneho balíku (bundle) a je nutné pracovať priamo so zdrojovým
kódom.
Pri vývoji aplikácie boli použité Python verzie 3.7.4 a pip 19.0.3 (čo je nástroj pre správu závislostí projektu).
Pred začatím používania je nutné nainštalovať všetky závislosti zo súboru \emph{requirements.txt}.
Tento súbor obsahuje informácie o balíčkoch a ich konkrétnych verziách použitých v projekte.
Pre odstránenie potenciálneho konfliktu už nainštalovanej verzie je odporúčané vytvoriť si tzv. virtuálne prostredie
(angl. virtual environment).
Jedná sa o vytvorenie priečinku, v ktorom sú obsiahnuté všetky informácie (interpreter, balíčky) týkajúce sa daných
požiadaviek projektu (najmä závislosti).
Takýmto spôsobom sa vytvorí izolované prostredie pre projekt, ktoré je nezávislé od potenciálne vytvoreného globálneho
prostredia.
Použitie aplikácie je napr.: \shellcmd{python ./src/\_\_main\_\_.py game}.

Predpis použitia je \shellcmd{python ./src/\_\_main\_\_.py [task] [parameters]} a aplikácia prijíma nasledovné
argumenty:
\begin{itemize}
    \item \shellcmd{task}: (slovensky \enquote{úloha}) vyjadruje analogicky úlohu, ktorá sa má vykonať.
    Možný je výber z nasledujúcich:
    \begin{itemize}
        \item \shellcmd{ann}: spustenie simulácie s trénovaním umelej neurónovej siete
        \item \shellcmd{minmax}: spustenie simulácie s využitím algoritmu minimax
        \item \shellcmd{analyze}: spustenie experimentov s rôznymi plochami
        \item \shellcmd{game}: spustenie experimentov s rôznymi modelmi
        \item \shellcmd{playground}: z názvu vyplýva, že je na programátorovi, čo sa spustí v rámci tohto príkazu, pričom sú
        dostupné všetky potrebné náležitosti, ktoré majú dostupné aj iné úlohy (napr. model ANN, model minimax-u, atď.)
        \item \shellcmd{store-ann}: natrénovanie a uloženie neurónovej siete
    \end{itemize}
    \item \shellcmd{\texttt{-{}-}size=[number]}: rozmer hracej plochy (resp. hracieho priestoru); ekvivalent $r$
    \item \shellcmd{\texttt{-{}-}to-win=[number]}: počet rovnakých, za sebou idúcich znakov potrebných pre výhru; ekvivalent $w$
    \item \shellcmd{\texttt{-{}-}dimension=[number]}: dimenzia (2 alebo 3); ekvivalent $d$
    \item \shellcmd{\texttt{-{}-}player=[X,O]}: preferovaný (cieľový) hráč
    \item \shellcmd{\texttt{-{}-}simulations=[number]}: počet simulácii hier (resp. veľkosť trénovacej množiny +
    veľkosť testovacej množiny)
    \item \shellcmd{\texttt{-{}-}validations=[number]}: počet validácii (slúži pre získanie reálnych štatistík)
    \item \shellcmd{\texttt{-{}-}max-depth=[number]}: maximálna veľkosť vyhľadávacieho stromu pri algoritme \emph{minimax};
    ekvivalent $l$
    \item \shellcmd{\texttt{-{}-}ann-model=[adaptive,large]}: zvolený model \emph{umelej neurónovej siete}
    \begin{itemize}
        \item \shellcmd{adaptive} je model popísaný v \autoref{subsec:algo-ann}
        \item \shellcmd{large} je model použitý pri experimentoch, popísaný v \autoref{subsec:experiments-board}
    \end{itemize}
\end{itemize}

\subsection{Príručky pre grafickú časť}\label{subsec:helpers-graphics}

\subsubsection{Programátorská príručka}
V rámci projektov unity je odporúčané držať sa verzie, v ktorej je projekt vyvíjaný (2019.2.17f1), pretože pri
zložitejších štruktúrach môže dôjsť ku konfliktom.
Pre otvorenie projektu vo vývojovom prostredí je nutné mať nainštalovanú túto verziu unity a aplikáciu
\enquote{unity editor}.
V tomto programe je možné upravovať vizuálnu stránku aplikácie, vytvárať moduly a meniť väzby medzi nimi.
Pre úpravu kódu v rámci jazyka C\# je možné použiť akékoľvek vývojové prostredie (skr. IDE, angl. integrated development
environment), ktoré podporuje tento jazyk.
Medzi najpoužívanejšie patria \emph{Microsoft Visual Studio} alebo \emph{JetBrains Rider}, v ktorom bola aplikácia
vyvíjaná.
Aplikáciu je možné skompilovať dvoma spôsobmi:
\begin{enumerate}
    \item odporúčaným spôsobom v danom IDE
    \item pomocou príkazového riadku.
    Ako príklad je možné uviesť kompilovanie v OS Windows:
    \shellcmd{"c:{\textbackslash}Program Files{\textbackslash}Unity{\textbackslash}Hub{\textbackslash}Editor{\textbackslash}2019.2.17f1{\textbackslash}Editor{\textbackslash}Unity.exe"\\ \-batchmode \-nographics \-executeMethod BuildScript.PerformBuild \-quit \\\-projectPath game \-logFile \-}
    Tento konkrétny zápis je použitý aj pri zostavení aplikácie v GitLab-CI .
\end{enumerate}

\subsubsection{Používateľská príručka}
Zostavený program je pripravený pre použitie aj na klasickom desktopovom počítači a sčasti aj v prostredí Cave.
Pri zapnutí programu sa zobrazí hlavné menu, kde sú dve možnosti: začať hru a ukončiť program (oba s anglickými
názvami).
Navigácia v menu je vykonávaná pomocou vertikálnej osi.
Zvolením možnosti ukončiť program sa aplikácia vypne, po zvolení možnosti začať hru je zobrazené ďalšie menu, kde si
používateľ volí parametre hry.
Konkrétne ide o veľkosť hracieho priestoru (3-8), počet za sebou idúcich znakov potrebných pre výhru a znak hráča, s
ktorým chce užívateľ hrať.
Pri voľbe parametrov sa výber $w$ automaticky zhora ohraničí na základe rozmeru hracieho poľa a teda
$w \in {3, \dots, r}$.
Parametre sa volia využitím horizontálnej osi.
Po vyplnení všetkých potrebných parametrov sa hra spustí.
Pomocou oboch osí je možné otáčať hrací priestor, pričom najbližšie políčko, na ktoré sa \enquote{pozerá kamera} je
vyznačené zelenou guľou.
V hre sa navyše využívajú navyše ešte 3 osi pre akčné tlačidlá:
\begin{enumerate}
    \item priblíženie v rámci hracieho priestoru
    \item oddialenie v rámci hracieho priestoru
    \item priradenie vlastného znaku na označené políčko
\end{enumerate}
Po priradení znaku sa automaticky vyplní aj jedno políčko protihráča, ktoré vyberie algoritmus umelej inteligencie.
V prípade, že jeden z hráčov vyhrá alebo nastane remíza s tým, že všetky políčka sú už obsadené, vypíše sa na obrazovku
informácia o stave hry a po 7 sekundách je uskutočnený návrat do hlavného menu.
\\
\\
Príručky sú jasne popísané, takže nech ktokoľvek sa rozhodne pokračovať v projekte, mal by vedieť ovládať obe časti
aplikácie pre svoje vlastné potreby.
