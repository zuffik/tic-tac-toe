\section{Umelá inteligencia}\label{sec:ai}

Od počiatku vzniku mechanických strojov, ktoré pomáhajú ľuďom pri svojej práci sa pri nich nepriamo niesol aj pojem
\enquote{umelá inteligencia}.
Medzi odbornou verejnosťou tento pojem stále nie je jednotný, no väčšina z nich má rovnakú myšlienku:
inteligencia vykonávaná strojmi (narozdiel od prirodzenej inteligencie, ktorú vykonávajú ľudia, či zvieratá).
Ak stroj (aj mechanický) dokáže niečo vykonať bez explicitného príkazu, je považovaný za inteligentný.

\subsection{Metódy}\label{subsec:ai-methods}

Umelá inteligencia (angl. artificial intelligence) je široká vedná disciplína, ktorá zahŕňa
\begin{itemize}
    \item strojové učenie
    \item textovú analýza
    \item analýzu reči
    \item prevod textu na reč
    \item expertné systémy
    \item plánovanie
    \item optimalizáciu
    \item robotiku
    \item analýzu obrazu
    \item a mnoho ďalšieho
\end{itemize}
Metódy niektorých skupín sa môžu prelínať (napr. na analýzu obrazu a na analýzu zvuku sa môžu použiť umelé neurónové
siete).
Dôsledkom toho je aj fakt, že niektoré metódy sa používajú viac a iné menej.
Využiteľnosť metód závisí teda najmä od problému, ktorý riešia.

Ak problémom je plánovanie trasy pre kuriérov, použije sa matematická optimalizácia.
Ukážkou môže byť ako americká spoločnosť UPS v amerických mestách, kde je systém ulíc riešený kolmým spôsobom, čo
najviac obmedzila zatáčanie kuriérov \emph{vľavo}.\cite{ups_optimization}
Dôvodom je najmä to, že keď kuriér odbočuje vľavo, musí dať prednosť protiidúcim vozidlám, pričom auto stojí na
križovatke naštartované.
Dôsledkom sú nasledovné vylepšenia:
\begin{enumerate}
    \item ušetrených takmer 40 miliónov litrov paliva ročne
    \item o 100 tisíc ton $CO_2$ menej
\end{enumerate}
Rovnaké vylepšenia by prinieslo odstavenie 21 tisíc áut.

Na druhú stranu, keď sa rieši problém rozpoznávania hlasových príkazov je nutné problém rozdeliť na 2 podproblémy:
\begin{enumerate}
    \item \emph{spracovanie zvuku}, napr. s využitím skrytých markovových modelov\cite{hmm}
    \item \emph{spracovanie textu}\cite{text_analysis}
\end{enumerate}

Je teda zjavné, že algoritmus, ktorý rozpoznáva význam textu nebude stačiť pre plánovanie trás kuriérov.
Nižšie sú popísané metódy, ktoré sú v rámcu umelej inteligencie využívané najlepšie.

\subsubsection{Heuristické metódy}

Heuristiky sú často používanou metódou vo vyhľadávacích algoritmoch.
Na rozdiel od exaktných algoritmov, ktoré prehľadávajú celý vyhľadávací priestor, heuristiky prehľadávajú len okolie
východiskového riešenia.
Ak napríklad je cieľom heuristiky nájsť minimum nejakej funkcie (nemusí byť zadaná analyticky) v $n$-rozmernom priestore,
dokáže táto metóda nájsť len lokálne minimum závislé na tom, kde sa nachádza východzie riešenie.
Príklad je uvedený pre funkciu
\begin{equation}
    f(x,y)=e^{\cos(x)+\sin(y)}\frac{x+y}{30}+\frac{5}{2}
\end{equation}
kde $x\in<-12,12>$, $y\in<-12,12>$.
\begin{figure}[H]
    \centering
    \includegraphics[width=0.8\textwidth]{images/heuristic.png}
    \caption{Heuristika a jej riešenia v závislosti od východzieho riešenia}
\end{figure}\label{figure:heuristic-method}
Pri nastavení východiskového bodu \textbf{A} alebo \textbf{B} nájde heuristika rovnaké lokálne minimum (čo sa môže javiť
ako globálne), no pre východzí bod \textbf{C} nájde algoritmus globálne minimum, čo ale nie je možné overiť bez
prehľadania celého priestoru riešení.
Z toho je možné usúdiť, že riešenie heuristiky nie je vždy optimálne a nikdy jeho optimálnosť nie je zaručená, ale na
druhú stranu sú heuristiky extrémne rýchle.
Heuristické metódy patria medzi najzákladnejšie techniky v rámci umelej inteligencie.

Ako ukážkový príklad praktického využitia heuristických metód je možné uviesť hľadanie najkratšej (alebo najrýchlejšej)
cesty medzi dvoma mestami v cestnej sieti Slovenska pre navigačné systémy alebo pohyb neovládaných postáv v hrách
(angl. non-player character, skr. NPC - akákoľvek postava v hre, ktorú neovláda človek).
Pre oba tieto príklady je možné použiť napríklad \emph{A* algoritmus}.

\subsubsection{Metóda podporných vektorov}

Anglicky support vector machine (skr. SVM) je metóda strojového učenia s učiteľom (reinforcement learning).
Cieľom tejto metódy je nájsť takú \emph{nadrovinu} v $n$-rozmernom vyhľadávacom priestore, ktorá vstupné dáta rozdelí
do dvoch podpriestorov.
Metóda je určená pre klasifikáciu a regresnú analýzu.
Ak je metóda konštruovaná ako optimalizačná úloha jej účelová funkcia vyzerá podobne ako nasledovná:
\begin{equation}
    \max \sum_{\forall i}{\sqrt{\sum_{\forall j}{(\vec{v}_j - x_{ij})^2}}}
\end{equation}
Kde $\vec{v}$ je hľadaný vektor a $\vec{v}_j$ sú jeho zložky a kde $x_i$ je vzorka zo vstupných dát a $x_{ij}$ sú jeho
zložky.
Keďže výraz $\sqrt{\sum_{\forall j}{(\vec{v}_j - x_{ij})^2}}$ vyjadruje najmenšiu vzdialenosť medzi vektorom a
vzorkou, dá sa povedať, že metóda hľadá takú nadrovinu, ktorá vyhľadávací priestor rozdelí čo najlepšie.
\begin{figure}[H]
    \centering
    \includegraphics[width=0.5\textwidth]{images/svm.png}
    \caption{Support vector machine}
\end{figure}\label{figure:svm}

Typickým príkladom použitia metódy podporných vektorov je rozdelenie e-mailovej komunikácie do dvoch skupín: relevantnú
a nevyžiadanú poštu, no podporné vektory sa dajú použiť aj na spracovanie obrazu, či textu.

\subsubsection{Iné metódy}

Pre umelú inteligenciu existuje veľké množstvo algoritmov a ich alternatív, z ktorých sú ešte často používané
\emph{umelé neurónové siete}, ktoré sú bližšie popísané v \autoref{subsec:algo-ann} a ako rozhodovacie pravidlo v
teórii hier je často využívaný algoritmus \emph{minimax} (viac v \autoref{subsec:algo-minmax}).
