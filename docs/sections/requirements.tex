\section{Analýza požiadaviek pre zvolené možnosti riešenia}\label{sec:requirements}

\subsection{Požiadavky pre implementáciu umelej inteligencie}\label{subsec:requirements-ai}

V tejto časti je nutné, aby vývojové prostriedky, teda programovací jazyk poprípade inštalovateľné súčasti jazyka, boli
schopné zvládnuť nasledovné požiadavky:
\begin{enumerate}
    \item vedieť jednoducho vytvoriť štruktúru umelej neurónovej siete a algoritmu minimax
    \item mali by byť optimalizované pre použitie s matematickými konštrukciami
    \item vedieť jednoducho prepojiť vytvorené algoritmy s herným prostredím (napr. cez súborový systém, sockety, \ldots)
\end{enumerate}
Veľkou výhodou je, ak má jazyk možnosti ako zabrániť duplicite kódu (napr. využitím polymorfie) a ak jazyk je v
súčasnosti udržiavaný komunitou programátorov.
Tieto požiadavky najlepšie spĺňajú\cite{best_ai_languages} aj tieto jazyky:

\subsubsection{Python}

Python je veľmi obľúbený jazyk medzi programátormi kvôli jeho jednoduchosti, netradičným programátorským
konštrukciám, ktoré skracujú a sprehľadňujú zdrojový kód aplikácií.
Poskytuje aj balíčky, ktoré pomáhajú s matematickými operáciami (napr. práca s maticami, funkciami, \dots).
Týchto balíčkov je veľké množstvo a je len na architektovi systému, ktorý si vyberie.
Pri výbere tohto jazyka zavážili aj skúsenosti autora tejto práce v jazyku python.

Tento jazyk je interpretovaný (tzn. nie kompilovaný) programovací jazyk vytvorený v roku 1991 programátorom
Guido van Rossum-om.\cite{lang_python}
Jeho dizajn je zameraný na \emph{čitateľnosť kódu} s využitím najmä "bielych znakov" (medzery, tabulátory) a odsadenie,
čo je signifikantný rozdiel oproti iným komerčným jazykom, ktoré používajú prevažne zátvorky \shellcmd{\{\dots\}} na
vyznačenie schémy kódu (pozn. "schéma kódu" je preložená z angl. scope).
Python nie je striktne typovaný jazyk, čo sa môže javiť ako jeho nevýhoda, no statické typovanie je možné jednoducho
pridať.
Je multi-paradigmový, čo znamená, že podporuje viacero prístupov k programovaniu.
Medzi tieto prístupy patrí: objektovo-orientované programovanie, štrukturálne programovanie, funkcionálne programovanie
apod.
Tento jazyk ovplyvnil aj rôzne ďalšie jazyky ako ECMAScript, Go, Kotlin, Swift\dots.

\subsubsection{R}

Programovací jazyk R je efektívny najmä kvôli práci so súbormi, poskytuje možnosti jednoduchej analýzy.
Tento programovací jazyk má ale zložitejšiu syntax, ktorá napriek skvelému spracovaniu pre umelú inteligenciu,
nespĺňala výhodu.
Tento jazyk je zameraný najmä na štatistické výpočty a prvý krát sa objavil v roku 1976.\cite{lang_r}
Jeho autorom je John Cambers a vyvinuli ho v tíme známej spoločnosti Bell Labs.
Podobne ako jazyky MatLab alebo APL vie aj tento programovací jazyk pracovať s maticami, vektormi a ponúka aj možnosti
pre prácu s relačnými databázami vo forme dátových rámcov (data frame).

Ako bolo spomenuté, R dobre pracuje so štatistickými modelmi.
Ak existuje na disku csv súbor (čiarkou oddelené hodnoty, angl. comma separated values), potom napríklad lineárnu
regresiu v tomto jazyku je možné zapísať nasledovne (y je závislá premenná, x sú nezávislé premenné):
\begin{lstlisting}[
  mathescape,
  columns=fullflexible,
  basicstyle=\fontfamily{lmvtt}\selectfont,
]
    data = read.csv('subor.csv')
    model <- lm(y~x1+x2+x3)
    summary(model)
\end{lstlisting}

\subsubsection{Lisp}

Názov vychádza z anglického \textbf{Lis}t \textbf{p}rocessing a je to rodina jazykov vytvorených v roku 1958.
Tento programovací jazyk (alebo rodina jazykov) je jeden z najstarších programovacích jazykov a vznikol pod rukou
John-a McCarthy-ho, ktorý sa zároveň považuje za otca umelej inteligencie.\cite{father_of_ai}
Starší je už len jazyk Fortran (o 1 rok).
Jazyk lisp v súčasnosti nie je udržiavaný komunitou natoľko, aby bol dostatočný pre túto prácu.
Pôvodne bol jazyk vyvinutý ako matematický zápis pre počítačové programy a rýchlo sa adaptoval ako jazyk pre
výskum umelej inteligencie.
Jazyk pracuje s tzv. S-výrazmi (S-expressions) a sú to v podstate volania funkcii s rôznymi argumentami.
Volanie funkcie \shellcmd{func} s argumentami \shellcmd{arg1, arg2, arg3, arg4} je zapísané ako:
\begin{lstlisting}[
  mathescape,
  columns=fullflexible,
  basicstyle=\fontfamily{lmvtt}\selectfont,
]
    (func arg1 arg2 arg3 arg4)
\end{lstlisting}

\subsection{Požiadavky pre implementáciu grafickej časti}\label{subsec:requirements-game}

Súčasťou práce je aj grafická časť (hra), ktorú používateľ ovláda v prostredí Cave.
Je teda nutné, aby bol zvolený grafický editor (resp. grafický engine), ktorý celú túto časť obsiahne a dokáže:
\begin{enumerate}
    \item kooperovať s implementáciou umelej inteligencie čo najjednoduchším spôsobom
    \item vytvárať scény pre prostredie Cave čo najjednoduchším spôsobom
    \item poskytovať voľne dostupný variant (licenciu) používania aplikácie
\end{enumerate}
Keďže grafické enginy používajú najmä herné štúdiá a individualisti tvoriaci hry s veľmi variabilným obsahom,
ukazovateľom pre výber editora môže byť okrem vypísaných aj jeho popularita.
Pri tvorbe je nutné dbať na kompatibilitu s prostredím Cave, no veľkou výhodou je možnosť kompilácie projektu
pre čo najväčší počet rôznych platforiem.
Práve na základe popularity\cite{best_3d_game_engines} boli porovnávané 3 najpoužívanejšie grafické enginy.

\subsubsection{Godot}

Herný engine, ktorý kladie dôraz na dve veci: jednoduchosť a škálovateľnosť.\cite{game_engine_godot}
Oproti ostatným herným enginom ponúka širokú podporu rôznych programovacích jazykov ako python, C\# alebo C++.
Ponúka grafický editor, ktorý zaberá na disku veľmi málo miesta, no funguje na mnohých komerčných operačných systémoch
ako MacOS, Windows a Linux.
Všetky tieto výhody by z tohto engine spravili najlepšieho kandidáta pre túto prácu, bohužiaľ tento prostriedok je na
trhu relatívne krátko (prvá verzia bola vydaná v roku 2007) a má teda malú podporu pre prostredie Cave.

Hlavným cieľom tohto grafického enginu je možnosť vytvorenia hry "from scratch" (teda úplne od základov) bez použitia
akýchkoľvek nástrojov okrem tých na tvorbu obsahu.
Je multiplatformový a ponúka zostavenie produktu pre 3 hlavné desktopové operačné systémy: Windows s rôznymi nástrojmi,
macOS, Linux, tak isto zostavenie pre mobilné operačné systémy Android, iOS a dokonca aj BlackBerry a v neposlednom
rade aj podporu pre HTML5, a WebAssembly, čo je v súčasnosti veľmi žiadané.

\subsubsection{Unity}

Unity je jeden z najstarších a zároveň najrozšírenejších herných enginov.\cite{game_engine_unity}
Ponúka podporu pre programovacie jazyky založené na jazyku C (teda C\# a C++).
Tento engine má, vďaka jeho dlhej prítomnosti na trhu, veľkú komunitu používateľov a tým pádom aj veľkú podporu tejto
komunity;
Unity má založené fórum s miliónmi komentárov a dobre vypracovanú dokumentáciu.
Silne a natívne podporuje umelú inteligenciu\cite{game_engine_unity_ml_agents} a používatelia vytvorili aj podporu pre
Cave prostredie\cite{game_engine_unity_kave}.
Pre tento engine bol vytvorený aj tzv. asset store, kde používatelia zdieľajú alebo predávajú svoje výtvory (ako napr.
modely pre hru, pozadia, textúry,\dots).

Unity vznikol v roku 2005 v spoločnosti Apple, Inc. ako herný engine vytvorený exkluzívne pre systém MacOS (vtedy OSX).
Tak isto ako Godot ponúka aj Unity možnosť zostavenia aplikácie do rôznych platforiem.
Sú to všetky platformy, ktoré ponúka Godot ale navyše je možné kompilovať kód pre PlayStation, Xbox, Oculus a iné
komerčné VR (virtuálna realita) platformy, ďalej pre Nintendo, Smart TV (ako napr. tvOS od Apple, Samsung Smart TV),
pre online herné platformy ako je Facebook Gameroom a rôzne AR platformy (Apple ARKit, Google ARCore).

Navyše pre potreby umelej inteligencie bol v rámci enginu Unity vyvinutý projekt Unity ML Agents.
Tento balík ponúka prostredie pre tréning inteligentných \emph{agentov} pre použitie v hrách.
Je teda zjavné, že táto možnosť toho ponúka oveľa viac ako Godot a teda tento herný engine je najlepší kandidát a to
aj z dôvodu znalostí a skúseností autora s týmto enginom.

\subsubsection{Unreal Engine}

Spolu s Unity je tento herný engine jedným z najstarších priekopníkov v hernom
priemysle.\cite{game_engine_unreal_engine}
Dôraz kladie najmä na kvalitu, čo je vidno aj na hrách postavených na v tomto prostriedku.
Dôsledkom tohto faktu je aj to, že Unreal Engine je optimalizovaný na jeden operačný systém (Windows) a je to veľká
nevýhoda pre iné operačné systémy.
Pracuje v jazyku Visual C++ a podporuje aj prepojenie s umelou inteligenciou no už menšiu podporu má pre Cave
prostredie.

Tento herný engine vznikol v spoločnosti Epic Games v roku 1998 ako súčasť hry \emph{Unreal}, čo bola akčná hra z
pohľadu prvej osoby (angl. first person shooter, skr. FPS).
Cieľom Unreal Engine je priniesť užívateľovi čo najpresnejšiu a najvernejšiu kópiu reálneho objektu alebo prostredia
a dá sa povedať, že všetky platformy, ktoré podporuje Unity, podporuje aj tento Unreal Engine.

\subsection{Ostatné požiadavky}\label{subsec:requirements-other}

K ostatným požiadavkám patrí najmä výpočtová sila pri trénovaní umelej neurónovej siete a prístup k zostavenému
prostrediu Cave.
Toto prostredie je dostupné na Žilinskej univerzite vo vedeckom parku, no z dôvodu nedostatočných zdrojov implementáciu
do Cave nebolo možné uskutočniť.
Ak by táto platforma bola dostupná, použiteľnosť unity by bolo nutné zabezpečiť len pre platformu Windows.
Toto obmedzenie existuje z dôvodu využitia knižnice \emph{GetReal3D}\cite{getreal3d}, ktorá je dostupná len pre Unity
a bola jedným z dôvodov, prečo bol zvolený Unity engine.
Knižnica zabezpečuje nielen zobrazovanie obrazu v 3D rozhraní, ale poskytuje aj možnosť premietnutia obrazu aj do Cave
prostredia.
Tento systém by bolo nutné pred použitím kalibrovať, aby poskytoval plnohodnotný užívateľský zážitok.
\\
\\
Na základe spomínaných výhod, nevýhod a skúseností bol zvolený programovací jazyk \emph{Python} kvôli jeho jednoduchému
prístupu k matematickým úlohám a tiež jeho ponúkaným balíčkom s hotovými riešeniami, spolu s herným enginom
\emph{Unity}, ktorý efektívne spolupracuje s umelou inteligenciou a tiež s prostredím Cave.

