\section{Analýza požiadaviek}\label{sec:requirements}

\subsection{Požiadavky pre časť s umelou inteligenciou}\label{subsec:requirements-ai}

V tejto časti je nutné, aby vývojové prostriedky, teda programovací jazyk poprípade inštalovateľné súčasti jazyka, boli
schopné zvládnuť nasledovné požiadavky:
\begin{enumerate}
    \item vedieť jednoducho vytvoriť štruktúru umelej neurónovej siete a algoritmu minimax
    \item mali by byť optimalizované pre použitie s matematickými konštrukciami
    \item vedieť jednoducho prepojiť vytvorené algoritmy s herným prostredím (napr. cez súborový systém, sockety, \ldots)
\end{enumerate}
Veľkou výhodou je, ak má jazyk možnosti ako zabrániť duplicite kódu (napr. využitím polymorfie) a ak jazyk je v
súčasnosti udržiavaný komunitou programátorov.
Tieto požiadavky najlepšie spĺňajú\cite{best_ai_languages} aj tieto jazyky:

\subsubsection{Python}
Python je veľmi obľúbený jazyk medzi programátormi kvôli jeho jednoduchosti, netradičným programátorským
konštrukciám, ktoré skracujú a sprehľadňujú zdrojový kód aplikácií.
Poskytuje aj balíčky, ktoré pomáhajú s matematickými operáciami (napr. práca s maticami, funkciami, \dots).
Týchto balíčkov je veľké množstvo a je len na autorovi, ktorý si vyberie.
Pri výbere tohto jazyka zavážili aj skúsentosti autora tejto práce v jazyku python.

\subsubsection{R}

Jazyk R je efektívny najmä pre kvôli práci so súbormi, poskytuje možnosti jednoduchej analýzy.
Tento programovací jazyk má ale zložitejšiu syntax, ktorá napriek skvelému spracovaniu pre umelú inteligenciu,
nespĺňala výhodu.

\subsubsection{Lisp}

Tento jazyk (alebo rodina jazykov) je jeden z najstarších programovacích jazykov.
Vznikol v roku 1958 pod rukou John McCarthy-ho, ktorý sa zároveň považuje za otca umelej
inteligencie.\cite{father_of_ai}
Jazyk lisp v súčasnosti nie je udržiavaný komunitou natoľko, aby bol dostatočný pre túto prácu.

\subsection{Požiadavky pre grafickú časť}\label{subsec:requirements-game}

Súčasťou práce je aj grafická časť (hra), ktorú používateľ ovláda v prostredí Cave.
Je teda nutné, aby bol zvolený grafický editor (resp. grafický engine), ktorý celú túto časť obsiahne a dokáže
\begin{enumerate}
    \item kooperovať s časťou s umelou inteligenciou čo najjednoduchším spôsobom
    \item vytvárať scény pre prostredie Cave čo najjednoduchším spôsobom
    \item poskytovať voľne dostupný variant používania aplikácie
\end{enumerate}
Keďže grafické enginy používajú najmä herné štúdiá a individualisti tvoriaci hry s veľmi variabilným obsahom,
ukazovateľom pre výber editora môže byť okrem vypísaných aj jeho popularita.
Práve na základe popularity\cite{best_3d_game_engines} boli porovnávané 3 najpoužívanejšie grafické enginy.

\subsubsection{Godot}
Herný engine, ktorý kladie dôraz na dve veci: jednoduchosť a škálovateľnosť.\cite{game_engine_godot}
Oproti ostatným herným enginom ponúka širokú podporu rôznych programovacích jazykov ako python, C\# alebo C++.
Ponúka grafický editor, ktorý zaberá na disku veľmi málo miesta, no funguje na mnohých komerčných operačných systémoch
ako MacOS, Windows a Linux.
Všetky tieto výhody by z tohto engine spravili najlepšieho kandidáta pre túto prácu, bohužiaľ tento prostriedok je na
trhu relatívne krátko a má teda malú podporu pre prostredie Cave.

\subsubsection{Unity}
Unity je jeden z najstarších a zároveň najpoužívanejších herných enginov.\cite{game_engine_unity}
Ponúka podporu pre programovacie jazyky založené na jazyku C (teda C\# a C++).
Tento engine má, vďaka jeho dlhej prítomnosti na trhu, veľkú komunitu používateľov a tým pádom aj veľkú podporu tejto
komunity;
Unity má založené fórum s miliónmi komentárov a dobre vytvorenú dokumentáciu.
Silne a natívne podporuje umelú inteligenciu\cite{game_engine_unity_ml_agents} a používatelia vytvorili aj podporu pre
Cave prostredie\cite{game_engine_unity_kave}.
Pre tento engine bol vytvorený aj tzv. asset store, kde používatelia zdieľajú alebo predávajú svoje výtvory (ako napr.
modely pre hru, pozadia, textúry,\dots.
Tento herný engine je najlepší kandidát aj z dôvodu znalostí a skúseností s týmto enginom.

\subsubsection{Unreal Engine}
Spolu s Unity je tento herný engine jedným z najstarších priekopníkov v hernom
priemysle.\cite{game_engine_unreal_engine}
Dôraz kladie najmä na kvalitu, čo je vidno aj na hrách postavených na v tomto prostriedku.
Dôsledkom tohto faktu je aj to, že Unreal Engine je optimalizovaný na jeden operačný systém (Windows) a je to veľká
nevýhoda pre iné operačné systémy.
Pracuje v jazyku Visual C++ a podporuje aj prepojenie s umelou inteligenciou no už menšiu podporu má pre Cave
prostredie.

\subsection{Ostatné požiadavky}\label{subsec:requirements-other}

K ostatným požiadavkam patrí najmä výpočtová sila pri trénovaní umelej neurónovéj siete a prístup k zostavenému
prostrediu Cave.
Toto prostredie je dostupné na Žilinskej univerzite vo vedeckom parku.
\\
\\
Na základe spomínaných výhod, nevýhod a skúseností bol zvolený programovací jazyk \emph{Python} s herným enginom
\emph{Unity}.

