\section{Analýza požiadaviek}\label{sec:requirements}

\subsection{Požiadavky pre časť s umelou inteligenciou}\label{subsec:requirements-ai}

V tejto časti je nutné, aby vývojové prostriedky, teda programovací jazyk poprípade inštalovateľné súčasti jazyka, boli
schopné zvládnuť nasledovné požiadavky:
\begin{enumerate}
    \item vedieť jednoducho vytvoriť štruktúru umelej neurónovej siete a algoritmu minimax
    \item mali by byť optimalizované pre použitie s matematickými konštrukciami
    \item vedieť jednoducho prepojiť vytvorené algoritmy s herným prostredím (napr. cez súborový systém, sockety, \ldots)
\end{enumerate}
Veľkou výhodou je, ak má jazyk možnosti ako zabrániť duplicite kódu (napr. využitím polymorfie) a ak jazyk je v
súčasnosti udržiavaný komunitou programátorov.
Tieto požiadavky najlepšie spĺňajú\cite{best_ai_languages} aj tieto jazyky:

\paragraph{Python} je veľmi obľúbený jazyk medzi programátormi kvôli jeho jednoduchosti, netradičným programátorským
konštrukciám, ktoré skracujú a sprehľadňujú zdrojový kód aplikácií.
Poskytuje aj balíčky, ktoré pomáhajú s matematickými operáciami (napr. práca s maticami, funkciami, \dots).
Týchto balíčkov je veľké množstvo a je len na autorovi, ktorý si vyberie.
Pri výbere tohto jazyka zavážili aj skúsentosti autora tejto práce v jazyku python.

\paragraph{R}
Jazyk R je efektívny najmä pre kvôli práci so súbormi, poskytuje možnosti jednoduchej analýzy.
Tento programovací jazyk má ale zložitejšiu syntax, ktorá napriek skvelému spracovaniu pre umelú inteligenciu,
nespĺňala výhodu.

\paragraph{Lisp}
Tento jazyk (alebo rodina jazykov) je jeden z najstarších programovacích jazykov.
Vznikol v roku 1958 pod rukou John McCarthy-ho, ktorý sa zároveň považuje za otca umelej
inteligencie.\cite{father_of_ai}
Jazyk lisp v súčasnosti nie je udržiavaný komunitou natoľko, aby bol dostatočný pre túto prácu.

\subsection{Požiadavky pre grafickú časť}\label{subsec:requirements-game}

Súčasťou práce je aj grafická časť (hra), ktorú používateľ ovláda v prostredí Cave.
Je teda nutné, aby bol zvolený grafický editor (resp. grafický engine), ktorý celú túto časť obsiahne a dokáže
\begin{enumerate}
    \item kooperovať s časťou s umelou inteligenciou čo najjednoduchším spôsobom
    \item vytvárať scény pre prostredie Cave čo najjednoduchším spôsobom
    \item poskytovať voľne dostupný variant používania aplikácie
\end{enumerate}
Keďže grafické enginy používajú najmä herné štúdiá a individualisti tvoriaci hry s veľmi variabilným obsahom,
ukazovateľom pre výber editora môže byť okrem vypísaných aj jeho popularita.
Práve na základe popularity\cite{best_3d_game_engines} boli porovnávané 3 najpoužívanejšie grafické enginy.

TODO
godot,unity,unreal
