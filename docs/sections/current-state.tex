\section{Analýza súčasného stavu}\label{sec:current-state-analysis}

V dnešnom svete je veľa technológií vytvorených, teoreticky popísaných a používaných.
Mnoho týchto technológií bolo vyvinutých konkrétne pre hernú scénu za účelom zlepšenia používateľského zážitku (angl.
user experience, skr. UX).
Či už sa jedná o jednoduchú hru, ako piškvorky, zložitejšiu hru, akou je napríklad šach, alebo komplexnú strategickú
hru, akou je napríklad StarCraft, pre každú je možné vytvoriť a študovať ich algoritmické vlastnosti.
Tieto vlastnosti sú už u prevažnej väčšiny hier už popísané a vo veľkej väčšine v nich figurujú rozhodovacie stromy.

Pri voľbe konkrétnej hry pre túto prácu boli zvolené piškvorky najmä kvôli nasledujúcim vlastnostiam.
Piškvorky sú veľmi jednoduchá hra, ktorá svojou jednoduchosťou zaujme každého (a pravdepodobne ju každý aspoň raz v
živote hral).
Zároveň ale ponúka veľa možností pre analýzu hracieho procesu a jeho algoritmizáciu.
Pretože aj dieťa dokáže nájsť v tejto hre rôzne vzory a rýchlo sa naučiť tento herný proces pomocou prirodzenej
inteligencie, je tento proces zaujímavý aj pre štúdium v rámci umelej inteligencie a matematickej interpretácie
problémov súvisiacich s touto hrou.

Fascinujúce je aj, že táto hra nie je zložitá, no stále komplexná a naskytá sa príležitosť pre preskúmanie algoritmov
súvisiacich s touto hrou \emph{do hĺbky}.
Zároveň je možné túto hru implementovať do ktoréhokoľvek prostredia a tieto algoritmy použiť v každom prostredí.
Ako bolo vyššie spomenuté, v súčasnosti je navrhnutých a preskúmaných mnoho metód pre túto hru najmä v zahraničnej
literatúre a niektoré z nich sú zaujímavé pre ich univerzálnosť (napr. algoritmus minimax sa dá využiť aj pri šachu).
