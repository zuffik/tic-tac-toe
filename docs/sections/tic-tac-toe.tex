\section{Piškvorky}\label{sec:tic-tac-toe}

Piškvorky sú hra známa pravdepodobne väčšine obyvateľov na tejto Zemi.
Je to hra, ktorá sa hrá s dvoma hráčmi prevažne na štvorcovej ploche.
Títo hráči majú pridelený svoj znak (buď \textbf{X} alebo \textbf{O}) sa pri hraní striedajú a vypĺňajú prázdne políčka
hracej plochy.
Víťazom sa stáva ten, ktorý má rovnaký počet znakov v riadku, stĺpci alebo diagonálne.
Predpokladaný vznik tejto hry je v roku 1300 pred naším letopočtom v starovekom Egypte.
Vtedy sa hrala vždy na ploche veľkosti $3 \times 3$ a ako hracie plochy slúžili strešné krytiny.
Na rozdiel od komplikovanejších hier sa kvôli jednoduchosti jej varianty od pôvodnej hry odlišujú len veľmi málo,
väčšinou sa jedná len o zmenu veľkosti poľa alebo množiny znakov, ktoré má hráč k dispozícii.
Jedným z variantov je napr. \emph{SOS}, kde majú hráči pre svoj ťah k dispozícii písmena \emph{S} a \emph{O} a snažia
sa vytvoriť v riadku, stĺpci alebo diagonálne slovo "SOS".
Kvôli svojej jednoduchosti sú piškvorky často hrané deťmi, ktoré ale rýchlo nájdu systém a rôzne opakujúce sa vzory
pri hraní.

Táto hra sa zároveň využíva aj v odvetví umelej inteligencie pri skúmaní vyhľadávacích stromov.
Pre piškvorky bolo vyvinutých mnoho exaktných aj heuristických algoritmov, z ktorých dva sú spomenuté v tejto práci.
V súčasnosti je piškvorkám venovaná len malá pozornosť, keďže algoritmy pre piškvorky sú už preskúmané do hĺbky.
Menej pozornosti sa venuje ale trojrozmerným piškvorkám.
S pridaním jedného rozmeru sa, samozrejme, väčšina z týchto algoritmov výrazne komplikuje a niektoré je nutné úplne
prerobiť.

\subsection{Označenie}\label{subsec:label}

Pre použitie je nutné zaviesť nasledujúce označenia:
\begin{itemize}
    \item $d$ - rozmer hry ($d \in \{2,3\}$)
    \item $r$ - rozmer hracieho poľa ($r \geq 3$)
    \item $n$ - počet všetkých buniek hracieho poľa (teda $r$ x $r$ na ploche alebo $r$ x $r$ x $r$ v priestore)
    \item $w$ - počet za sebou idúcich znakov potrebných pre výhru ($3 \leq w \leq r$)
    \item $b_{ij}$ resp. $b_{ijk}$ - $d$-rozmerná matica vyjadrujúca hodnotu políčka v $i$-tom riadku a $j$-tom
    stĺpci (resp. $k$-tu hodnotu v $i$-tom riadku a $j$-tom stĺpci)
\end{itemize}

\subsection{Pravidlá}\label{subsec:rules}

\begin{enumerate}
    \item Každý hráč má pridelený znak \textbf{X} alebo \textbf{O}
    \item Začína hráč s označením \textbf{X}
    \item Hráči sa pri hre striedajú (\textbf{X}, \textbf{O}, \textbf{X}, \textbf{O}, \ldots)
    \item Vyhráva hráč, ktorý:
    \begin{enumerate}
        \item má v ktoromkoľvek riadku aspoň $w$ za sebou idúcich rovnakých znakov
        \item má v ktoromkoľvek stĺpci aspoň $w$ za sebou idúcich rovnakých znakov
        \item má diagonálne v ktoromkoľvek smere aspoň $w$ za sebou idúcich rovnakých znakov
    \end{enumerate}
    \item Pre jednoduchosť je možné uvažovať rozmery (pokiaľ nie je uvedené inak):
    \begin{enumerate}
        \item $3 \times 3$ pre plochu s počtom znakov potrebných pre výhru 3 ($d = 2$, $r = 3$, $w = 3$)
        \item $3 \times 3 \times 3$ pre priestor s počtom znakov potrebných pre výhru 3 ($d = 3$, $r = 3$, $w = 3$)
    \end{enumerate}
\end{enumerate}

\subsection{Úloha}\label{subsec:task}

Cieľom práce je ponúknuť hráčovi najlepší možný ťah, ktorý môže pri hraní aktuálneho kola vykonať.
Ak je daný rozmer 3x3 tak hru je možné ukončiť nasledovnými spôsobmi:\cite{number_of_wins}
\begin{itemize}
    \item remíza: \emph{46 080 spôsobov}
    \item výhra prvého hráča (\textbf{X}): \emph{131 184 spôsobov}
    \item výhra druhého hráča: (\textbf{O}): \emph{77 904 spôsobov}
\end{itemize}

Z toho vyplýva, že priestor, v ktorom sa prehľadáva riešenie je konečný (ohraničený).
S pridaním jedného rozmeru len v ploche (4x4) nebolo možné prerátať tieto počty (ani po 1h nebol výsledok známy).

