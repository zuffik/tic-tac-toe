\subsection{Piškvorky}\label{subsec:tic-tac-toe}

Piškvorky sú hra známa pravdepodobne väčšine obyvateľov tejto Zeme.
Je to hra, ktorá sa hrá s dvoma hráčmi prevažne na štvorcovej ploche.
Títo hráči, ktorí majú pridelený svoj znak (buď \textbf{X} alebo \textbf{O}), sa pri hraní striedajú a vypĺňajú prázdne
políčka hracej plochy svojím znakom.
Víťazom sa stáva ten, ktorý má rovnaký počet znakov v riadku, stĺpci alebo diagonálne.
Predpokladaný vznik tejto hry je v roku 1300 pred naším letopočtom v starovekom
Egypte.\cite{tic_tac_toe_first_appearance}
Vtedy sa hrala vždy na ploche veľkosti $3 \times 3$ a ako hracie plochy slúžili strešné krytiny.
Na rozdiel od komplikovanejších hier sa kvôli jednoduchosti jej varianty od pôvodnej hry odlišujú len veľmi
málo,\cite{tic_tac_toe_variants} väčšinou sa jedná len o zmenu veľkosti poľa alebo množiny znakov, ktoré má hráč k
dispozícii.
Jedným z variantov je napr. \emph{SOS}, v ktorom majú hráči pre svoj ťah k dispozícii písmena \emph{S} a \emph{O} a
snažia sa vytvoriť v riadku, stĺpci alebo diagonálne slovo "SOS".\cite{tic_tac_toe_sos}
V praxi túto hru, kvôli svojej jednoduchosti, hrajú najmä deti, ktoré ale pri hraní rýchlo nájdu systém a rôzne
opakujúce sa vzory.

Táto hra sa zároveň využíva aj v odvetví umelej inteligencie pri skúmaní vyhľadávacích stromov.
Pre piškvorky bolo vyvinutých mnoho exaktných aj heuristických algoritmov, z ktorých dva sú spomenuté v tejto práci.
V súčasnosti je piškvorkám venovaná len malá pozornosť, keďže algoritmy pre piškvorky sú už preskúmané do hĺbky.
Menej pozornosti sa ale venuje trojrozmerným piškvorkám.
S pridaním jedného rozmeru sa, samozrejme, väčšina z týchto algoritmov výrazne komplikuje a niektoré je nutné úplne
prerobiť.

\subsubsection{Označenie}

Pre použitie je nutné zaviesť nasledujúce označenia:
\begin{itemize}
    \item $d$ - rozmer hry - 2D alebo 3D ($d \in \{2,3\}$)
    \item $r$ - rozmer hracieho poľa v danom rozmere ($r \geq 3$)
    \item $n$ - počet všetkých buniek hracieho poľa (teda $r \times r$ na ploche alebo $r \times r \times r$ v
    priestore)
    \item $w$ - počet za sebou idúcich znakov potrebných pre výhru ($3 \leq w \leq r$)
    \item $b_{ij}$ resp. $b_{ijk}$ - $d$-rozmerná matica (štvorcová matica resp. tenzor) vyjadrujúca hodnotu políčka v
    $i$-tom riadku a $j$-tom stĺpci (resp. $k$-tu hodnotu v $i$-tom riadku a $j$-tom stĺpci)
\end{itemize}

\subsubsection{Pravidlá}

\begin{enumerate}
    \item Každý hráč má pridelený znak \textbf{X} alebo \textbf{O}
    \item Začína hráč s označením \textbf{X}
    \item Hráči sa pri hre striedajú (\textbf{X}, \textbf{O}, \textbf{X}, \textbf{O}, \ldots)
    \item Vyhráva hráč, ktorý:
    \begin{enumerate}
        \item má v ktoromkoľvek riadku aspoň $w$ za sebou idúcich rovnakých znakov
        \item má v ktoromkoľvek stĺpci aspoň $w$ za sebou idúcich rovnakých znakov
        \item má diagonálne v ktoromkoľvek smere aspoň $w$ za sebou idúcich rovnakých znakov
    \end{enumerate}
    \item Pre jednoduchosť je možné uvažovať rozmery (pokiaľ nie je uvedené inak):
    \begin{enumerate}
        \item $3 \times 3$ pre plochu s počtom znakov potrebných pre výhru 3 ($d = 2$, $r = 3$, $w = 3$)
        \item $4 \times 4$ pre plochu s počtom znakov potrebných pre výhru 3 ($d = 2$, $r = 4$, $w = 3$)
        \item $3 \times 3 \times 3$ pre priestor s počtom znakov potrebných pre výhru 3 ($d = 3$, $r = 3$, $w = 3$)
        \item $4 \times 4 \times 4$ pre priestor s počtom znakov potrebných pre výhru 3 ($d = 3$, $r = 4$, $w = 3$)
    \end{enumerate}
    \item hracia plocha je vždy štvorec alebo kocka, resp. $r$ je rovnaké vo všetkých rozmeroch
\end{enumerate}

\subsubsection{Zložitosť hry}

\paragraph{Teoretická zložitosť}

V piškvorkách sú políčka, ktoré sú reprezentované troma stavmi: \textbf{X}, \textbf{O} alebo prázdne.
Pre porovnanie, sú nižšie vypísané počty kombinácii pre všetky stavy hry v niektorých rozmeroch.
Počty sú počítané takým spôsobom, ktorý reprezentuje skutočnú hru, tzn. že sa hráči pri hre striedajú a počet voľných
políčok sa znižuje vždy o 1.
Teoreticky túto hodnotu reprezentuje hodnota $r^d!$.
\begin{itemize}
    \item $\mathbf{3 \times 3}$: $362\ 880$
    \item $\mathbf{4 \times 4}$: $\approx 2.10^{13}$
    \item $\mathbf{10 \times 10}$: $\approx 9.10^{157}$
    \item $\mathbf{100 \times 100}$: $\approx 2.10^{35\ 659}$
    \item $\mathbf{3 \times 3 \times 3}$: $10^{28}$
    \item $\mathbf{4 \times 4 \times 4}$: $\approx 10^{89}$
    \item $\mathbf{10 \times 10 \times 10}$: $\approx 4.10^{2\ 567}$
    \item $\mathbf{100 \times 100 \times 100}$: $\approx 2.10^{5\ 565\ 708}$
\end{itemize}
Z toho vyplýva, že priestor, v ktorom sa prehľadáva riešenie je konečný (ohraničený).
Preto boli vybrané algoritmy, ktoré vedeli pracovať s \emph{ohraničeným} priestorom riešení.
Je možné si všimnúť, že počty kombinácií sú extrémne vysoké na to, aby ich zvládla komerčná výpočtová technika.

\paragraph{Zložitosť v praxi}

Ak je daný rozmer pre hru $3 \times 3$ tak túto hru je možné ukončiť nasledovnými spôsobmi:\cite{number_of_wins}
\begin{itemize}
    \item remíza: \emph{46 080 spôsobov}
    \item výhra prvého hráča (\textbf{X}): \emph{131 184 spôsobov}
    \item výhra druhého hráča: (\textbf{O}): \emph{77 904 spôsobov}
\end{itemize}
S pridaním jedného rozmeru len v ploche ($4 \times 4$) nebolo možné prerátať tieto počty a ani po 1 hodine výsledok
nebol známy.