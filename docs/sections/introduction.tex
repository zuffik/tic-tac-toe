\section*{Úvod}

Už od dávnych čias ľudia vynaliezajú rôzne nástroje, aby si zjednodušili život, prácu a spríjemnili voľný čas.
Od kolesa, ktoré pomáhalo pri transporte objektov, cez kalkulačku až po súčasnú virtuálnu realitu.
Ak navyše tieto nástroje dokážu rozmýšľať sami za seba a nepotrebujú zásah človeka, je to veľká výhoda.
V dnešnej dobe stačí povedať do smartfónu príkaz a technológie vyriešia našu požiadavku ak sú toho schopné.
Takýmto spôsobom sa dajú napríklad zasvietiť svetlá v dome, zistiť informácie o počasí a mnoho ďalšieho.
Všetko je to dôsledkom vývinu umelej inteligencie, ktorej najväčší pokrok bol zaznamenaný len v druhej polovici
minulého storočia.
V súčasnosti je táto technológia už na pokročilej úrovni, pre väčšinu projektov teda stačí, aby bola zvolená metóda,
ktorá daný problém vyrieši čo najefektívnejšie, čo vo výsledku znamená zvyšovať presnosť riešenia a znižovať zdroje
potrebné pre vykonanie výpočtov.

Ako sa zlepšovali technológie, zlepšovali sa aj spôsoby zobrazenia reálnych objektov vo virtuálnom svete.
Prostriedky virtuálnej reality je dnes možné mať aj na svojom smartfóne.
Existujú ale riešenia, ktoré vo svojich mobilných telefónoch mať nemôžme, ale sú vo svojom výkone efektívne.
Jedným z nich je aj prostredie Cave, ktoré možno nie všetci poznáme.
Niekoľko projektorov, 3D okuliare a ovládač sú základným stavebným prvkom tohto prostredia a vďaka tomu sa osoba
stojaca v tomto prostredí cíti akoby bola priamo na mieste, ktoré zobrazuje.
