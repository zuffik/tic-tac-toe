\section*{Záver}

Preskúmané metódy ponúkajú svoje výhody aj nevýhody ako napríklad výpočtový čas, či efektivita riešenia.
V hre piškvorky bol ako algoritmus umelej inteligencie použitý kompromis medzi rýchlosťou a presnosťou: umelá neurónová
sieť.
Táto metóda sa vyrovná akejkoľvek heuristickej metóde, no ponúka jednoduché a rýchle spracovanie a predikciu.
Výsledky sú, v prípade porovnaní rôznych algoritmov a rôznych modelov, uspokojivé.
Navyše je v aplikácii implementovaný algoritmus zisťovania víťaza pre 3D piškvorky s variabilným počtom znakov
potrebných pre výhru.

Pretože z dôvodu nedostatku zdrojov nebolo možné spracovať aplikáciu pre prostredie Cave, je pripravená aspoň pre
použitie v klasických osobných počítačoch.
Herná časť bola naprogramovaná v hernom engine unity, ktorý poskytuje jednoduchú tvorbu rôznych 2D a 3D prostredí.

Táto práca ukázala aký potenciál má technológia Cave v spojení s umelou inteligenciou.
Pokračovanie v podobných projektoch by mohlo viesť napríklad aj k vytvoreniu prostredí, ktoré by svojím spracovaním
simulovali verejne nedostupné, no zaujímavé miesta.
Prostredníctvom tejto technológie by sa tieto miesta priblíži k bežným občanom.
Ak navyše by sa táto technológia spojila napríklad s hlasovými pokynmi, stačilo by zadať príkaz a používateľ by sa
ocitol úplne na inom mieste.
